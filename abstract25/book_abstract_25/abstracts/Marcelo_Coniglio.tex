\begin{abstract_online}
{On Morgado and Sette's Implicative hyperlattices as models of da Costa logic $C_\omega$}
{Marcelo E. Coniglio}
{University of Campinas, Brazil}
{}
        
\begin{abstract}
Jos\'e Morgado introduced in 1962 an original and interesting notion of hyperlattices, that he called {\em reticuloides}. In his Master's thesis defended in 1971 (and supervised by Newton da Costa), Antonio M. Sette proposed a novel notion of implicative hyperlattices (here called SIHLs) based on Morgado's hyperlattices. He also extended SIHLs by adding an unary hyperoperator, obtaining a class of hyperalgebras (here called SHC$\omega$s) which correspond to da Costa's algebras for $C\omega$, being so a suitable  semantics for the well-known  da Costa's logic  $C_\omega$.
In this talk we show new interesting results about Morgado's hyperlattices and  Sette's implicative hyperlattices. In particular, a natural characterization of SIHLs in purely (hyper)lattice-theoretic terms will be  proved, showing that, as in the algebraic case, they are distributive. We also introduce a class of swap structures, a special class of hyperalgebras over the signature of $C_\omega$ naturally induced by implicative lattices. It is proven that these swap structures are indeed SHC$\omega$s. Finally, it is proven that the class of  SHC$\omega$s, as well as the above mentioned class of swap structures, characterize the logic $C_\omega$. We argue that Morgado's hyperlattices constitute a very natural notion of hyperlattices, generalizing the algebraic case (based on partial orders) to hyperalgebras,  by moving to preorders.
This is a joint work with  Ana C. Golzio and Kaique M. Roberto.
\end{abstract}
    \end{abstract_online}