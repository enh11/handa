\begin{abstract_online}
{Analysis of Weak Associativity in Some  Algebraic Hypercompositional Structures that Represent Dismutation Reactions}
{Temitope Gbolahan JAIYEOLA, Kehinde Gabriel ILORI  and Oyeyemi Oluwaseyi OYEBOLA}
{\\ Department of Mathematics, University of Lagos, Akoka, Nigeria}
{}
        
\begin{abstract}
In this paper, some chemical systems of Tin (Sn), Indium (In) and Vanadium (V) which are represented by algebraic hypercompositional structures $(S_{Sn},\oplus)$, $(S_{In},\oplus)$ and $(S_{V},\oplus)$ were studied. The analyses of their algebraic properties and the probabilities of elements in dismutation reactions were carried out with the aid of computer codes in Python  programming language. It was shown that in the dismutation reactions, the left nuclear ($N_\lambda$)-probability, middle nuclear ($N_\mu$)-probability and right nuclear ($N_\rho$)-probability for each of the algebraic hypercompositional structures $(S_{Sn},\oplus)$, $(S_{In},\oplus)$ and $(S_{V},\oplus)$ is less than 1.000. This implies that, $(S_{Sn},\oplus)$, $(S_{In},\oplus)$ and $(S_{V},\oplus)$ are non-associative algebraic hypercompositional structures . Also, from the results obtained for FLEX-probability, it was shown that, $(S_{Sn},\oplus)$, $(S_{In},\oplus)$ and $(S_{V},\oplus)$ have flexible elements because the values of their FLEX-probabilities are 1.000 each. Hence, $(S_{Sn},\oplus)$, $(S_{In},\oplus)$ and $(S_{V},\oplus)$ are flexible. Overall,  $(S_{V},\oplus)$ exhibited the lowest measure of weak-associativity, $(S_{Sn},\oplus)$ exhibited lower measure of weak-associativity, and $(S_{In},\oplus)$ exhibited a low measure of weak-associativity. 
\end{abstract}
    \end{abstract_online}