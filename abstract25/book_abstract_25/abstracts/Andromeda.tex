\begin{abstract_online}
{Fuzzy aspects in biological inheritance}
{Andromeda Pătrașcu Sonea}
{Ion Ionescu de la Brad Iasi University of Life Sciences}
{}
        
\begin{abstract}
Genetics is the science that studies the heredity and variability of organisms. Genetics explains the mechanisms of recording, modifying, and transmitting hereditary information from generation to generation. Biologist and mathematician Gregor Mendel is considered the founder and father of genetics. He developed the theory of hereditary factors, according to which a specific material particle determines each character of the organism, called a hereditary factor (gene). The study of hyperstructures in this context is necessary because the results of the experiments in the case of the monohybrid cross in the pea plant, the dihybrid cross in the pea plant, the inheritance in the Four-o’ clock plant, and so on describe hypergroups.  
This paper will emphasize the fuzzy function associated with the classes formed by the phenotypes resulting from the simple dominance (dihybrid cross, trihybrid cross, 4-hybrid cross). We have chosen to study this because it has been observed that there is a link between the fuzzy function and the number of resulting phenotypes.
\end{abstract}
    \end{abstract_online}