\begin{abstract_online}
{AI Tools and the Exploration of Hypercompositional Structures in Algebra}
{Charalampos Tsitouras}
{National and Kapodistrian University of Athens, Greece}
{}
        
\begin{abstract}
The recent proliferation of artificial intelligence tools, particularly large language models such as ChatGPT, offers powerful new methods for exploring algebraic systems defined by hypercompositions. Hypercompositional structures, such as hypergroups, hypergroupoids, hyperfields, hyperrings etc, extend classical algebraic notions by allowing operations to produce sets of outcomes rather than single values. These systems are foundational in generalized algebra and have applications in automata theory, fuzzy systems, and non-classical logics.
This lecture presents how AI tools can assist in the construction, enumeration, classification, and analysis of hypercompositional structures. Using case studies—such as the enumeration of hypergroups of small orders, verification of structural properties like associativity and reproductivity, and isomorphism detection—we demonstrate how symbolic computation environments (e.g., Mathematica) combined with natural language interfaces (e.g., ChatGPT) can streamline both research and pedagogical tasks. In particular, we emphasize how various results—such as those of Massouros and Tsitouras on the enumeration of hypergroups—can be interactively verified and further explored with the support of AI tools.
The presentation will conclude with a discussion on the prospects of integrating machine learning with algebraic hypercompositional structures, potentially automating the discovery of structural invariants and classifying large families of algebraic structures.

\end{abstract}
    \end{abstract_online}