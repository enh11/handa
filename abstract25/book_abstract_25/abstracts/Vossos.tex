\begin{abstract_online}
{Can Hypercompositional Algebra contribute to the resolution of the unsolved problems of modern Physics?}
{Spyridon Vossos and Elias Vossos}
{National and Kapodistrian University of Athens, Greece}    
{} 
\begin{abstract}
Hypercompositional Algebra (HA), a generalization of classical algebraic systems through multi-valued operations, can offer a novel mathematical lens for addressing some of the unresolved challenges in modern physics. By extending traditional structures such as groups, rings, and fields into hypercompositional structures—where binary operations yield sets rather than single elements—HA introduces tools capable of modeling ambiguity, non-determinism, and complex systemic interactions.
This presentation explores the potential of HA to inform theoretical developments in areas such as Newtonian physics, Special Relativity (SR), General Relativity (GR), Quantum mechanics, Quantum field theory, Symmetry breaking, and Quantum gravity. More specifically, we examine if the Linear Transformations:
\begin{itemize}
\item Closed Isometric Complex Boost in Isotropic Complex Spacetime and
\item Open Isometric Generalized Real Boost in Isotropic Real Spacetime,
\end{itemize}
which led to Generalized Special Relativity (which Unifies the Newtonian Physics and Einsteinian Relativity Theory), could also lead to a new Hypergroup (which includes both the Galilean group and Lorentz group).

Moreover, we examine if the Scale factor a(t) in the FLRW metric (which is used in the Study of the expansion of Universe), could also be studied via the Hypercompositional Algebra.
\end{abstract}
\end{abstract_online}