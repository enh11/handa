\begin{abstract_online}
  {On the class of Transposition Hypergroups}
  {Christos Massouros and Gerasimos Massouros}
  {National and Kapodistrian University of Athens, Greece,\\Hellenic Open University, Greece}
  {}
\begin{abstract}
The introduction of a third axiom in hypergroups—the transposition axiom—has opened up a vast area of research within hypercompositional algebra.
The class of transposition hypergroups includes structures such as quasicanonical hypergroups (or polygroups), canonical hypergroups, join spaces (or join hypergroups), fortified transposition hypergroups, fortified join hypergroups, and others.
Some of these structures are used in constructing more complex hypercompositional systems—such as hyperfields, hyperrings, hypermodules, etc.—whose additive component is a canonical hypergroup, or equivalently, a commutative transposition hypergroup with scalar identity.
Several open problems remain concerning the development and analysis of analogous structures in which the additive part is a different type of transposition hypergroup, e.g., a fortified join hypergroup.
\end{abstract}
    \end{abstract_online}
