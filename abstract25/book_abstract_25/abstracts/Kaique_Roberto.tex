\begin{abstract_online}
{Hyper swap structures}
{Kaique M. Roberto}
{Faculdade Einstein, São Paulo, Brazil}
{}
        
\begin{abstract}
Non-deterministic matrices (a.k.a. Nmatrices) constitute a natural generalization of logical matrices, in which an algebra of truth-values is replaced by an hyperalgebra. Valuations over Nmatrices  \lq\lq pick\rq\rq, for every formula, a single element of the set of values produced by the hyperoperator associated to the main connective of the formula, when applied to the value assigned to its immediate subformulas. This notion was formalized in 2001 by Avron and Lev, but it was already used by  several authors, notoriously by Ivlev in the 1970s in the context of non-normal modal logics. In some cases, finite-valued Nmatrices provide decision procedures for logics that cannot be characterized by a single finite-valued matrix. In 2016 Carnielli and Coniglio introduced the notion of swap structures, which are hyperalgebras formed by elements of a given algebra (typically a Boolean algebra or a Heyting algebra) representing semantical states  of a formula in terms of formulas of the language of the underlying algebra. The hyperoperators are defined by imposing relational conditions to (some of) the coordinates of the input states. Swap structures can be seen as non-deterministic twist structures.
In this talk we introduce the notion of hyper swap structures, which are swap structures based on an hyperalgebra. This natural generalization of swap structures allows to construct equivalences between categories of hyperalgebras, as happens with twist structures in the deterministic (algebraic) case. As an illustrative example, it will be shown that the categories of hyperalgebras for da Costa's logic $C_\omega$ and the category of Sette's implicative hyperlattices are equivalent.
This is a joint work with Marcelo E. Coniglio and Ana C. Golzio.
\end{abstract}
\end{abstract_online}