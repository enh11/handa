\begin{abstract_online}
{Weakly free multialgebras}
{Guilherme Toledo}
{Department of Computer Science, Bar Ilan University, Israel \\
Center for Logic, Epistemology and History of Science, University of Campinas, Brazil}
{}
\begin{abstract}
In abstract algebraic logic, many systems, such as those paraconsistent logics taking inspiration from da Costa's hierarchy, are not algebraizable by even the broadest standard methodologies, as that of Blok and Pigozzi. However, these logics can be semantically characterized by means of non-deterministic algebraic structures such as Nmatrices, RNmatrices and swap structures. These structures are based on multialgebras, which generalize algebras by allowing the result of an operation to assume a non-empty set of values. This leads to an interest in exploring the foundations of multialgebras applied to the study of logic systems.

It is well known from universal algebra that, for every signature, there exist algebras over
which are absolutely free, meaning that they do not satisfy any identities or, alternatively, satisfy the universal mapping property for the class of algebras. Furthermore, once we fix a cardinality of the generating set, they are, up to isomorphisms, unique, and equal to algebras of terms (or propositional formulas, in the context of logic). Equivalently, the forgetful functor, from the category of algebras to Set, has a left adjoint. This result does not extend to multialgebras. Not only multialgebras satisfying the universal mapping property do not exist, but the forgetful functor, from the category of multialgebras to Set, does not have a left adjoint.

In this paper we generalize, in a natural way, algebras of terms to multialgebras of terms, whose family of submultialgebras enjoys many properties of the former. One example is that, to every pair consisting of a function, from a submultialgebra of a multialgebra of terms to another multialgebra, and a collection of choices (which selects how a homomorphism approaches indeterminacies), there corresponds a unique homomorphism, what resembles the universal mapping property. Another example is that the multialgebras of terms are generated by a set that may be viewed as a strong basis, which we call the ground of the multialgebra. Submultialgebras of multialgebras of terms are what we call weakly free multialgebras. Finally, with these definitions at hand, we offer a simple proof that multialgebras with the universal mapping property for the class of all multialgebras do not exist and that does not have a left adjoint.
\end{abstract}
    \end{abstract_online}