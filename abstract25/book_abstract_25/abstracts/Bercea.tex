\begin{abstract_online}
{Construction of finite fields and hyperfields}
{Bianca-Liana Bercea}
{Ovidius University from Constanta}
{}
\begin{abstract} 
%\textbf{Definition.}
%A commutative ring $(K, +, \cdot)$ with the property that any nonzero element $x$ in $K$ is
%invertible, i.e. exists $x^{-1}\in K$ such that $x\cdot x^{-1} = 1$ is called a field.
%
%\textbf{Example.}
%\begin{itemize}
%\item $\Q$, $\R$, $\C$ are numerical fields.
%\item The ring of integers $\Z$ is not a field because not all nonzero elements are invertible
%\item Let's consider $n\in\N^*$. Then, the ring $\Z_n$ is a field if and only if $n$ is a prime number.
%\end{itemize}
We can ask ourselves how we can construct new examples of finite fields. Analyzing the field of
complex numbers and the fields $\Z_p$ with the prime number $p$, we will notice that we can construct
other examples of fields in the following way: we take a commutative ring A in which Euclid’s
algorithm can be applied, consider p a prime element, and take the set of residue classes upon
division by $p$, denoted $A_p$. Then $A_p$ will have the canonical field structure. Using these finite fields, we can construct certain finite hyperfields.
\end{abstract} 
\end{abstract_online}